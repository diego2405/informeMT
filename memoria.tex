\documentclass[11pt]{report}

\usepackage[utf8]{inputenc}
\usepackage[breaklinks]{hyperref}
\usepackage[spanish]{babel}
\usepackage{array}
\usepackage{amssymb}
\usepackage{amsmath}
\usepackage{amsfonts}
\usepackage{amsthm}
\usepackage{cite}
\usepackage{graphicx}
\usepackage{subcaption}
\usepackage{cleveref}
\usepackage{geometry}
\usepackage{multirow}
\usepackage{url}
\usepackage{tikz}
\usepackage{setspace} % Espaciado de líneas
\usepackage{tocloft}
\usepackage{subfiles} % varios archivos
\usepackage[ruled,vlined,spanish]{algorithm2e} %algoritmo
\graphicspath{{imagenes/}{../imagenes/}}
\usepackage[draft,color=gray!50,linecolor=black]{todonotes} % notas \todo[inline]{}
\captionsetup[subfigure]{subrefformat=simple,labelformat=simple}
\renewcommand\thesubfigure{(\alph{subfigure})}

\theoremstyle{definition}
\newtheorem{definition}{Definición}[section]
 \geometry{
 a4paper,
 total={210mm,297mm},
 left=25mm,
 right=20mm,
 top=25mm,
 bottom=20mm,
 }
\def\checkmark{\tikz\fill[scale=0.4](0,.35) -- (.25,0) -- (1,.7) -- (.25,.15) -- cycle;}
\pagestyle{headings}

\begin{document}

%\documentclass[../memoria.tex]{subfiles}

\begin{titlepage}
\thispagestyle{empty}
\begin{tabular*}{\textwidth}{l c@{\extracolsep{\fill}} r}
  UNIVERSIDAD DE CONCEPCIÓN & & \emph{Profesor Patrocinante:}\\
  Facultad de Ingeniería & & John Atkinson Abutridy\\
  Departamento de Ing. Civil Informática y & & \\
  Ciencias de la Computación & & \emph{Comisión:}\\
  & & Ma. Angélica Pinninghoff Junemann\\
  & & Javier Vidal Valenzuela

\end{tabular*}
{
\centering
\Large

~\vspace{\fill}

{\huge 
DESARROLLO DE UN PROTOTIPO DE SISTEMA DE RE-ID DE PERSONAS PARA VIDEO-VIGILANCIA
%DESARROLLO DE UN PROTOTIPO DE SISTEMA DE RE-ID DE PERSONAS PARA UNA APLICACIÓN DE VIDEO-VIGILANCIA
%Desarrollar un prototipo de sistema Re-ID de personas para vigilancia
%DESARROLLO DE UN MODELO DE COMPARACIÓN DE DESCRIPTORES PARA RE-IDENTIFICAR PERSONAS
}

\vspace{2cm}

{\LARGE
DIEGO A. REYES MOLINA
}

\vspace{2cm}
Informe de Memoria de Título\\
Para optar al Título de\\[1em]
Ingeniero Civil Informático

\vspace{\fill}

Noviembre 2015

}

\end{titlepage}
 %portada
\tableofcontents %índice de contenidos
\pagebreak
\listoftables %índice de tablas

\onehalfspace
\chapter{Introducción}
\subfile{secciones/introduccion}

%\chapter{Métodos de Re-ID (\emph{trabajo relacionado})}
%\subfile{secciones/trabajorelacionado}

\chapter{Sistemas de Re-ID}
\subfile{secciones/marcoteorico}

\chapter{Re-ID utilizando puntos de interés} %cambiar título
\subfile{secciones/modelo}

\chapter{Experimentos y Resultados}
\subfile{secciones/experimentos}

%\chapter{Conclusiones}
%\subfile{secciones/conclusiones}

%\chapter*{Apéndice}
%\subfile{secciones/apendice}

\bibliographystyle{acm}
\bibliography{bibliografia/bibliografia}
%\bibliography{bibliografia/intro,bibliografia/marcoteorico,bibliografia/trabajorelacionado,bibliografia/estudios,bibliografia/apariencia,bibliografia/dml}
\end{document}
